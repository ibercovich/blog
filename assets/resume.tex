\documentclass[a4paper,10pt]{article}
\usepackage[utf8]{inputenc}
\usepackage{geometry}
\usepackage{enumitem}
\usepackage{hyperref}

% Adjust the page margins
\geometry{left=2.5cm, top=1.5cm, right=2.5cm, bottom=2.0cm}

\begin{document}

% Header
\noindent
\textbf{\Large Ivan Bercovich} \\
\href{mailto:ibercovich@gmail.com}{ibercovich@gmail.com} \\
\vspace{0cm}

% Overview
\section*{Overview}
\noindent
I am experienced in leading engineering/product teams in the 100-200 person range. I also have general management experience at Amazon where I co-led the office in Santa Barbara, and as a CEO with HeyTutor. I've largely worked with structured data, knowledge graphs, and NLP. I believe innovation can emerge anywhere in the organization, and I dedicate energy and resources to nurture creativity. Within my teams, managers are given latitude to take calculated risks and engage in experimentation. Recently, I’ve been focused on venture investing. My experience allows me to select strong technical teams, identify invention risks, and complement the executive teams at portfolio companies as they scale.

% Experience
\section*{Experience}

\noindent
% ScOp Venture Partners
\textbf{ScOp Venture Partners} | 2021 - Present \\
I founded ScOp with 3 partners to leverage my operating experience across a portfolio of opportunities. Our first fund was \$50M over 21 investments, and we focus on post-revenue companies close to the \$1M milestone. Depending on the needs of the founders, I might act as fractional CTO, sit on the board, or provide ad-hoc advice. Among my partners, I focus primarily on technical discovery and diligence, and general operating support (product/tech strategy, recruiting, org building, etc).

\vspace{5pt} % Add a blank line

\noindent
% HeyTutor
\textbf{HeyTutor} | 2022 - 2023 \\
While working as an investor, I joined as interim-CEO for a year to help HeyTutor's transition from startup to growth business. Alongside the executive team, we adopted stricter procedures and executed meaningful organizational changes for a workforce of 1000+ employees and tutors. We grew the business by 5x, to \$36M in revenue, and became profitable.

\vspace{5pt} % Add a blank line

\noindent
% Unwrap
\textbf{Unwrap.ai} | 2020 - 2021 \\
I was the co-founder and first CEO of Unwrap, a startup I incubated with the Allen Institute for AI. We focused on making sense of customer feedback at scale. I built the initial models, hired the first engineer, and raised seed capital, but ultimately decided to pursue investing full time.

\vspace{5pt} % Add a blank line

\noindent
% Amazon Alexa
\textbf{Amazon Alexa} | 2017 - 2020 \\
Amazon acquired our company, Graphiq, to improve Alexa's open domain question understanding and answering (Q\&A) capabilities. At the time, 2017, Alexa was behind competitors in the Q\&A space, but assisted by the Graphiq acquisition, Alexa caught up*\footnote{*Amazon 2018 shareholder letter: "Last year, we improved Alexa’s ability to understand requests and answer questions by more than 20\%, while adding billions of facts to make Alexa more knowledgeable than ever."}  to be near or above the best-in-class assistant in most knowledge categories. When Graphiq's CEO left, I became the GM for the team of ~150 software developers and knowledge engineers (KE), who jointly built the technology and knowledge graph (KG) supporting Alexa's semantic Q\&A capabilities. By the time I ended my tenure in 2020, the Graphiq technology was handling most Q\&A across all languages and locales in which Alexa was present. We developed solutions for natural language processing, knowledge retrieval, and language generation. In parallel, we built a number of internal tools and workflows that enabled our KEs to define complex knowledge representations and ingest billions of facts to build one of the most comprehensive KGs.

\vspace{5pt} % Add a blank line

\noindent
% Graphiq
\textbf{Graphiq} | 2011 - 2017 \\
I started as a software developer, then managed the 40-person engineering team as VP, and eventually succeeded the CEO. Graphiq focused on collecting, structuring, and connecting the world’s data — billions of entities and hundreds of billions of facts. We began by building stand-alone vertical search engines for specific categories, but upon understanding the value of interconnected information, we progressively developed a complex ontology and KG. Our website and embeddable widgets allowed people to thoroughly research thousands of topics on one intuitive interface. Launched in late 2010, Graphiq quickly became a leading research engine with 40+ million monthly visits, and 400 million monthly impressions of our KG across the internet. In 2016, as voice assistants became better known, we invested in novel approaches using natural language processing for KG retrieval, leading to multiple acquisition offers.

\vspace{5pt} % Add a blank line

\noindent
% Cisco Systems
\textbf{Cisco Systems} | 2009 \\
I worked as a UX engineer on Cisco Quad - an attempt to build a social network for the enterprise.

% Education
\section*{Education}

\noindent
% University of California, Santa Barbara
\textbf{University of California, Santa Barbara (unfinished)} | 2010 \\
Ph.D. in Statistics and Applied Probability with emphasis in Financial Mathematics.

\vspace{5pt} % Add a blank line

\noindent
% University of Massachusetts Amherst
\textbf{University of Massachusetts Amherst} | 2005 - 2009 \\
BS in Electrical Engineering and Mathematics, Summa Cum Laude. GPA 3.95/4.00 \\
Departmental Honors with greatest distinction and top 2\% (or 2 std dev).

\end{document}
