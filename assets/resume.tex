\documentclass[a4paper,10pt]{article}
\usepackage[utf8]{inputenc}
\usepackage{geometry}
\usepackage{enumitem}
\usepackage{hyperref}

% Adjust the page margins
\geometry{left=2.5cm, top=1.5cm, right=2.5cm, bottom=2.0cm}

\begin{document}

% Header
\noindent
\textbf{\Large Ivan Bercovich} \\
Santa Barbara, California \\
\href{mailto:ibercovich@gmail.com}{ibercovich@gmail.com} \\
\vspace{0.3cm}

% Overview
\section*{Overview}
\noindent
I am experienced in leading engineering/product teams in the 100-200 person range. I also have GM experience at Amazon where I led the office in Santa Barbara, following our acquisition. My technical work has gravitated around structured data, knowledge graphs, and NLP. I’m biased towards simple solutions, particularly when complexity is the result of chasing the wrong trend. I believe innovation can emerge anywhere in the organization, and I dedicate energy and resources to nurture creativity. I believe leaders should be given latitude to take calculated risks and engage in experimentation regarding both process and objectives. I measure a leader’s long-term success by the accomplishments of others whom they’ve led. Recently, I’ve been focused on venture investing. My experience allows me to select strong technical teams, identify invention risks, and complement the executive teams at portfolio companies as they scale.

% Experience
\section*{Experience}

\noindent
% ScOp Venture Partners
\textbf{ScOp Venture Partners} | 2020 - Present \\
I founded ScOp with 3 partners to leverage my operating experience across a portfolio of opportunities. Our first fund is \$50M over 21 investments, and we focus on post-revenue companies close to the \$1M milestone. Depending on the needs of the founders, I might act as fractional CTO, sit on the board, or provide ad-hoc advice. Among my partners, I focus primarily on technical discovery and diligence, and general operating support for our founders (product/tech strategy, recruiting, org building, etc).

\vspace{5pt} % Add a blank line

\noindent
% Amazon Alexa
\textbf{Amazon Alexa} | 2017 - 2020 \\
Amazon acquired our company, Graphiq, to improve Alexa's open domain question understanding and answering (Q\&A) capabilities. At the time, 2017, Alexa was behind competitors in the Q\&A space, but assisted by the Graphiq acquisition, Alexa caught up*\footnote{*Amazon 2018 shareholder letter: "Last year, we improved Alexa’s ability to understand requests and answer questions by more than 20\%, while adding billions of facts to make Alexa more knowledgeable than ever."}  to be near or above the best-in-class assistant in most knowledge categories. When Graphiq's CEO Kevin O'Connor left, I became the GM for the entire team of ~150 software developers and knowledge engineers, who jointly built the technology and knowledge graph supporting Alexa's semantic Q\&A capabilities. By the time I ended my tenure in 2020, the Graphiq technology was answering most Q\&A questions across all languages (English, German, Japanese, Spanish, French, Portuguese, Hindi) and locales in which Alexa was present. The technology our team built takes a transcribed input from automatic speech recognition and handles all the natural language processing, knowledge retrieval, and natural language generation, and then hands off the final response to a text to speech system. In parallel, we built a number of internal tools and workflows that enabled our knowledge engineers to define complex knowledge representation via a proprietary ontology and ingested billions of facts to build one of the most comprehensive knowledge graphs (KG) in the world. Additionally, we made some of this technology available to customers through Alexa's Structured Knowledge Skills offering (in beta).

\vspace{5pt} % Add a blank line

\noindent
% Graphiq
\textbf{Graphiq} | 2011 - 2017 \\
I started as a software developer, then managed the 40-person engineering team as VP, and eventually succeeded the CEO. Graphiq focused on collecting, structuring, and connecting the world’s data — billions of entities and hundreds of billions of facts. We began by building stand-alone vertical search engines for specific categories, but upon understanding the value of interconnected information, we progressively developed a complex ontology and knowledge graph. Our website and embeddable widgets allowed people to thoroughly research thousands of topics on one intuitive interface. Launched in late 2010, Graphiq quickly became a leading research engine with 40+ million monthly visits, and 400 million monthly impressions of our Knowledge Graph across the internet. In 2016, as voice assistants became better known, we invested in novel approaches using natural language processing for KG retrieval, leading to multiple acquisition offers.

\vspace{5pt} % Add a blank line

\noindent
% Cisco Systems
\textbf{Cisco Systems} | 2009 \\
I worked as a UX engineer on Cisco Quad - an attempt to build a social network for the enterprise.

% Education
\section*{Education}

\noindent
% University of California, Santa Barbara
\textbf{University of California, Santa Barbara (unfinished)} | 2010 \\
Ph.D. in Statistics and Applied Probability with emphasis in Financial Mathematics.

\vspace{5pt} % Add a blank line

\noindent
% University of Massachusetts Amherst
\textbf{University of Massachusetts Amherst} | 2005 - 2009 \\
BS in Electrical Engineering, Summa Cum Laude. GPA 3.95/4.00 \\
BS in Mathematics with individual concentration in Mathematics, Summa Cum Laude. GPA 3.97/4.00 \\
Commonwealth College Departmental Honors with greatest distinction and top 2\% of class (e.g. 2 standard deviation).

\end{document}
